% This is samplepaper.tex, a sample chapter demonstrating the
% LLNCS macro package for Springer Computer Science proceedings;
% Version 2.20 of 2017/10/04
%
\documentclass[runningheads]{llncs}
%
\usepackage{pgfgantt}

\usepackage{graphicx}
\usepackage[strings]{underscore}
\usepackage[hyphens]{url}
\usepackage{ragged2e}
\usepackage{tabularx}
\usepackage[ruled,vlined]{algorithm2e}
\usepackage{listings}
\usepackage{caption}
\usepackage{subcaption}
\usepackage{layouts}

% set custom section numbers depth
\setcounter{secnumdepth}{5}
\usepackage{listings}
\usepackage{xcolor}
\usepackage{amsmath}

% define styling for code highlighting
\definecolor{codegreen}{rgb}{0,0.6,0}
\definecolor{codegray}{rgb}{0.5,0.5,0.5}
\definecolor{codepurple}{rgb}{0.58,0,0.82}
\definecolor{backcolour}{rgb}{0.95,0.95,0.92}

\lstdefinestyle{mystyle}{
	backgroundcolor=\color{backcolour},   
	commentstyle=\color{codegreen},
	keywordstyle=\color{magenta},
	numberstyle=\tiny\color{codegray},
	stringstyle=\color{codepurple},
	basicstyle=\ttfamily\footnotesize,
	breakatwhitespace=false,         
	breaklines=true,                 
	captionpos=b,                    
	keepspaces=true,                 
	numbers=left,                    
	numbersep=5pt,                  
	showspaces=false,                
	showstringspaces=false,
	showtabs=false,                  
	tabsize=2,
	frame=single
}
\lstset{style=mystyle}

\setlength{\parindent}{0in}
% Used for displaying a sample figure. If possible, figure files should
% be included in EPS format.
%
% If you use the hyperref package, please uncomment the following line
% to display URLs in blue roman font according to Springer's eBook style:
% \renewcommand\UrlFont{\color{blue}\rmfamily}
\renewcommand\refname{Literaturangaben}

\begin{document}
%
\title{Hauptprojekt \\~\\ Konzeption und prototypische Implementierung einer verteilten Autoscaling-Architektur für Cloud-Bursting mit Container-as-a-Service}
%
\titlerunning{Hauptprojekt}
% If the paper title is too long for the running head, you can set
% an abbreviated paper title here
%
\author{Christian F. Bargmann}
%
% \authorrunning{F. Author et al.}
% First names are abbreviated in the running head.
% If there are more than two authors, 'et al.' is used.
%
\institute{Hamburg University of Applied Sciences, Berliner Tor 5, 20099 Hamburg, Germany \\
	\email{christian.bargmann@haw-hamburg.de} \\
	\url{https://www.haw-hamburg.de}}
%
\maketitle              % typeset the header of the contribution
%
\begin{abstract} Cloud-Bursting ist ein Betriebsmodell, bei dem eine Anwendung in einer privaten Cloudumgebung oder einem Rechenzentrum betrieben wird, jedoch automatisch Ressourcen einer öffentlichen Cloud provisioniert werden, wenn die Nachfrage nach Rechenkapazität ansteigt. Die Entscheidung, anhand welcher Kriterien Ressourcen in eine öffentliche Cloud ausgelagert werden, wie sich die provisonierten Ressourcen in die lokal betriebene Infrastruktur integrieren lassen und welches Servicemodell des Cloud-Providers verwendet werden soll, ist in den vergangenen Jahren in den Fokus aktueller Forschung gerückt. Dieses Hauptprojekt konzeptioniert, implementiert und evaluiert eine verteilte Autoscaling-Architektur für Cloud-Bursting in Hybrid-Clouds, die automatisiert Ress\-ourcen auf Basis von Metriken als Container-as-a-Service bei mehreren Cloud-Service-Providern provisionieren und skalieren kann.
		
	\keywords{Cloud Bursting \and Cloud Computing \and Autoscaling \and \newline Container \and Software Architektur}
\end{abstract}
%
%
%
	
\section{Einleitung} \label{motivation}
	
Cloud-Bursting ist ein Betriebsmodell, bei dem eine Anwendung in einer privaten Cloudumgebung oder einem Rechenzentrum betrieben wird, jedoch automatisch Ressourcen einer öffentlichen Cloud provisioniert werden, wenn die Nachfrage nach Rechenkapazität ansteigt. Bei diesem hybriden Betriebsmodell wird die eigene Infrastruktur vollständig genutzt, so dass eine Anzahl von Servern in eigener Verantwortung und Kontrolle betrieben werden kann. Gleichzeitig besteht bei Lastspitzen die Möglichkeit, die Anwendung ganz oder teilweise in eine externe, öffentliche Cloud zu verlagern. \\
	
Die wichtigste Technologie bei Cloud-Computing zum Hosten und Verwalten von Anwendungen ist die Virtualisierung. Traditionell wird Hardware-Level-Virtualisierung, auch bekannt als Hypervisor-basierte Virtualisierung, zur Verwaltung von virtuellen Maschinen (VMs) in Cloud-Rechenzentren verwendet. Ein grosser Fortschritt in der Virtualisierungstechnologie ist die Containerisierung von Anwendungen, die auch als Virtualisierung auf Betriebssystemebene bekannt ist. Aufgrund der besseren Portabilität, des geringen Ressourcenbedarfs und der einfachen Skalierbarkeit im Vergleich zur VM-basierten Virtualisierung, hat Containerisierung in den letzten Jahren deutlich Popularitaet gewonnen. Containerisierung eignet sich für die Verwaltung von Microservices, da sie das schnelle Starten und Beenden von Containern und somit schnelle Skalierbarkeit unterstützt, während die VM-basierte Virtualisierung vergleichsweise mehr Zeit für das Starten und Beenden der VM benötigt \cite{al-dhuraibi_elasticity_2018}, \cite{abdullah_containers_2019}. Viele Cloud-Service-Provider (CSPs) bieten Container-as-a-Service Angebote an, mit denen sich Container in einer Cloud-Umgebung betreiben lassen, ohne darunterliegende Infrastruktur selbst verwalten zu muessen. \\

Fuer viele Unternhemen ist es unterschiedlichsten Gruenden nicht moeglich, ihre gesamte Infrastruktur zu einem CSP zu verlagern. Trotzdem bietet es sich situationsbedingt an, Teile der Infrastruktur bei Lastspiten auszulagern und kurzzeitig die eigenen Rechenkapazitaeten durch das Hinzuziehen von Ressourcen eines CSP zu steigern.
Trotz moderner As-a-Service-Angebote, ist die Umsetzung eines Hybrid-Cloud-Ansatzes fuer viele Unternehmen ein nicht zu unterschaetzender Aufwand. Die Integration zwischen der eigenen Infrastruktur und den Schnittstellen der CSPs, sowie die Wahl von geeigneten Skalierungsmethoden, ohne dabei zu stark an eine bestimmte Cloud-Plattform gekoppelt zu sein, ist fuer viele Unternehmen eine grosse Herausforderung.  \\

In dieser Arbeit wird eine verteilte Autoscaling-Architektur für Cloud-Bursting in Hybrid-Clouds vorgestellt, die automatisiert Ressourcen auf Basis von Metriken als Container-as-a-Service bei einem CSP provisionieren und horizontal skalieren kann. Die vorgestellte Architektur ist in der Lage, Anwendungen ueber mehrere Cloud-Umgebungen hinweg zu skalieren und provisionierte Ressourcen von mehreren Anbietern parallel in die eigene lokale Infrastruktur einzubinden. Fuer die automatische Skalierung von internen Ressourcen bei Lastspitzen zu CSPs werden Metriken aus externen Monitoring-Systemen genutzt, statt eigene Metriken zu implementieren. Im Vergleich zu anderen Ansaetzen (vgl. Abschnitt \ref{work}.), setzt die hier praesentierte Loesung nicht vorraus, das lokal ein Container-Cluster betrieben wird, was den Einsatz dieser Loesung attraktiv fuer Anwender macht, die kein Container-Cluster betreiben koennen, aber trotzdem ein Autoscaling fuer lokal betriebene Anwendungen benoetigen. Auch muessen lokal betriebene Anwendungen nicht modifiziert werden, um durch die vorgestellte Architektur skaliert zu werden. Die Komponenten der Autoscaling-Architektur koennen sowohl On-Premise, als auch extern bei einem CSP betrieben werden. \\

Desweiteren wird ein Prototyp mit einer reaktiven Skalierungstechnik, die den Lebenszyklus von Ressourcen beruecksichtigt, implementiert und das Verhalten der Autoscaling-Architektur unter verschiedenen Lastszenarien evaluiert.

	
\section{Related Work} \label{work}
	
Danayi et al. \cite{danayi_opencot_2019} präsentieren in ihrer Arbeit eine Implementierung eines Cloud Managers für das Internet of Things. Die vorgestellte Architektur nutzt Function-as-a-Service (FaaS) als Servicemodell, um bei Lastspitzen einzelne Funktionen zu einem öffentlichen Cloud-Provider auszulagern. Hierbei wird ein zentraler Cloud-Broker als Gateway-Komponente verwendet, der sämtliche Funktionsanfragen entgegennimmt und Logik zur Skalierung von Ressourcen an eine Controller-Komponente delegiert. Das Framework nutzt und speichert eigene Metriken für die Skalierung von Ressourcen und ist auf die Programmiersprache Python beschränkt. Für das Internet of Things bietet sich FaaS als ein geeignetes Servicemodell an, jedoch kann FaaS für viele Anwendungszenarien zu feingranular sein. \\
	
Biswas et al. \cite{biswas_auto-scaling_2015} erläutern ein Framework mit einer reaktiven Skalierungsmethode, um SLOs-basierte und nicht SLOs-basierte Anfragen zu verarbeiten. In einer fortführenden Arbeit werden hybride Skalierungsmethoden anhand von Kostenanalysen präsentiert \cite{biswas_hybrid_2017}. Hierbei wird der Cloud Broker bei einem Drittunternehmen eingesetzt, welches im Bedarfsfall Ressourcen für die private Cloud des Klienten bei einem Public-Cloud-Anbieter provisioniert. Der Klient hat keine Kontrolle über den eingesetzten Broker des Drittunternehmens, ebenso ist der Broker stark gekoppelt an die Schnittstellen des Public-Cloud-Providers. \\
	
Ye et al. \cite{ye_auto-scaling_2017} stellen eine Autoscaler-Architektur mit hybriden Skalierungsmethoden vor, die in einer containerbasierten, privaten Cloudumgebung eingesetzt werden können und sowohl das aktuelle Anfragevolumen, als auch Service-Level-Objectives (SLOs) für die geeignete Skalierung von Ressourcen berücksichtigen. Zwar werden hier automatisch Ressourcen bei Lastspitzen in der eigenen private Cloudumgebung skaliert, allerdings werden  keine externen Ressourcen hinzugezogen. \\

Gandhi et al. \cite{gandhi_adaptive_nodate} praesentieren in ihrer Arbeit eine Architektur fuer einen vollstaendig automatisierten Cloud-Service, der proaktiv deployte Anwendungen in einer Cloud-Infrastruktur auf Basis einer Kombination aus Anwendungs- und systemspezifischen Metriken skalieren kann. Als zu skalierende Instanzen werden keine Container, sondern virtuelle Maschinen verwendet. Die Architektur ist in der Lage, ueber die Schnittstellen der Cloud-Infrastruktur dynamisch VMs zu provisionieren und einzubinden, sowie Ressourcenkapazitaeten anzupassen. Um ein Autoscaling der lokalen Infrastruktur durchzufuehren, ist der Einsatz von on-premise Cloud-Infrastruktur und IaaS-Diensten notwendig. \\
	
In einem Artikel stellt Chandra \cite{chandra_cloud_2020} eine Möglichkeit für Cloud-Bursting mithilfe der Kubernetes-Erweiterungen Virtual Kubelet\footnote{https://github.com/virtual-kubelet/virtual-kubelet} und KIP\footnote{https://github.com/elotl/kip} vor. Die vorgestellte Methode setzt den Betrieb eines eigenen Clusters in der privaten Cloudumgebung voraus und nutzt Autoscaling-Funktionalitäten des Container-Orchestrators Kubernetes für die horizontale Skalierung von Container-Instanzen. Es wird vorausgesetzt, dass die zu skalierenden Instanzen zuvor als ein Deployment im Cluster angelegt worden sind. Weiterhin beschreibt Mennig \cite{mennig_cloud_2020} einen Multicluster-Ansatz mit dem Einsatz von Service Meshes, um Cloud Bursting umzusetzen. Hier wird Cluster-as-a-Service als Servicemodell beim Public-Cloud-Provider genutzt, um Compute Cluster zu provisionieren und diese in die on-premise betriebene Cluster-Infrastruktur einzubinden. Ein Service-Mesh wird für die Cross-Cluster-Kommunikation von Ressourcen verwendet, gemeinsam mit einem Multicloud-Scheduler um Workloads bei Lastspitzen über mehrere Cluster zu verteilen. Auch hier wird jedoch der Betrieb einer eigenen Cloud-Infrastruktur vorausgesetzt, ebenso wird Expertenwissen benötigt für die Installation und Wartung der Cluster-Erweiterungen, was für viele Unternehmen ein nicht zu unterschätzender Aufwand ist.
	
\subsection{Provisionierung von Ressourcen}

In verteilten Umgebungen wie Clouds, ist eine geeignete Skalierung von Ressourcen wichtig. Eine Unterskalierung kann zu einer vollstaendigen Auslastung der verfuegbaren Ressourcen führen, was zu langsamen Antwortzeiten oder einer großen Anzahl abgelehnter Anfragen führt.  Eine Ueberskalierung wiederum, verursacht nicht ausgelastete Ressourcen, wodurch Zusatzkosten für ungenutzte Rechenleistung entstehen. \\

Die Entscheidung über die Allokierung oder Freigabe von Ressourcen in Clouds, basiert in der Regel auf überwachten oder vorhergesagten Low-Level-Performance-Indikatoren (z.B. Auslastung von Ressourcen wie CPU, Speicher, Netzwerkbandbreite) oder High-Level-Indikatoren (z.B. Antwortzeit, Anfragerate). Weitere Metriken um ein Quality of Service (QoS) zu gewaehrleisten, werden auch in [17] und [18] vorgeschlagen. Zusammenfassend ist das Ziel dieser Arbeiten, bestimmte Werte von ueberwachten Indikatoren innerhalb  festgelegter Thresholds zu halten und bei Abweichung, Skalierungsmechanismen auszuloesen. Eine automatische Skalierung von Ressourcen kann reaktiv durchgefuehrt werden, zum Beispiel als Folge von Änderungen der Arbeitslast, oder proaktiv, d.h. anhand Vorhersagen von Änderungen der Arbeitslast. In mehreren Arbeiten werden auch Kombinationen der beiden Ansätze vorgeschlagen \cite{ali-eldin_adaptive_2012}, \cite{gandhi_adaptive_nodate}, \cite{fernandez_autoscaling_2014}, \cite{bouabdallah_use_2016}. Desweiteren gibt es Arbeiten wie \cite{moreno-vozmediano_efficient_2019}, \cite{iqbal_unsupervised_2016} und \cite{bu_reinforcement_2009}, in denen maschinelles Lernen und analytische Verfahren fuer eine praediktive Autoskalierung vorgestellt werden. \\

In dieser Arbeit wird eine reaktive Skalierungsstrategie vorgestellt, implementiert und in Abschnitt \ref{skalierungstechnik} im Detail erlaeutert. Verwendet werden vom Administrator festgelegte Regeln in Form von SLOs zur Einaltung bestimmter Thresholds.  \\
	
\section{Architekturübersicht} \label{Architekturübersicht}
	
In diesem Abschnitt wird auf die entwickelte Autoscaler-Architektur beschrieben. Zunaechst wird ein ueberblick ueber den Anwendungskontext gegeben und das Datenmodell vorgestellt. Anschliessend wird auf die einzelnen Komponenten der Architektur im Detail eingegangen und die Skalierungstechnik fuer die Allokierung und Freigabe von Ressourcen erlaeutert.
	
\subsection{Architektur}

Die Systemarchitekur des verteilten Autoscalers fuer Cloud-Bursting in Hybrid-Clouds ist in Fig. \ref{systemarchitektur} dargestellt. Abgebildet sind sowohl die interne on-premise, als auch externe Infrastruktur des CSPs. Der Autoscaler selbst besteht auf mehreren Teilanwendungen, die jeweils in einem eigenen Prozess gestartet werden und gemeinsam die automatische Skalierung von lokalen Anwendungen umsetzen.
	
\begin{figure}
	\centering
	\includegraphics[width=1.0\linewidth,scale=1.0]{images/context.png}
	\caption{Systemarchitektur und Kontextdarstellung der Autoscaling-Architektur}
	\label{systemarchitektur}
\end{figure}

Vorraussetzung fuer das automatische Skalieren einer lokal bereitgestellten Anwendung ist, dass diese Metriken zur Laufzeit exportiert und durch ein Monitoring-Tool ueberwacht wird. Ausserdem muss die zu skalierende Anwendung oder ein deploybarer Teil der Anwendung als ausfuehrbarer Container in einer Container-Registry vorliegen. \\

Kernelement der Architektur ist die Control-Plane-Komponente (vgl. \ref{control_plane}), welche in regelmaessigen Intervallen die von einem Administrator festgelegten SLOs gegenueber dem Monitoring-Tool evaluiert und die Allokierung oder Freigabe von Ressourcen berechnet. Ueber eine Webschnittstelle rufen die Agent Komponenten ueber ein Polling-Ansatz regelmaessig den Status ueber zu startende oder zu terminierende Instanzen zu einem Zeitpunkt ab. Im Vergleich zur Control-Plane sind die Agents statuslos und setzen den durch den Autoscaler berechneten Bedarf an Instanzen zu einem Zeitpunkt bei einem CSP um. Fuer jeden CSP zu dem Instanzen bei einer automatischen Skalierung ausgelagert werden sollen, wird genau ein Agent deployed. Weitere CSPs koennen ueber die Implementierung eines Interfaces angebunden werden. Bei jeder Statusveraenderung einer Instanz (vgl. Status) schreibt der Agent diese zurueck an die Control-Plane. Zwischen dem aufrufenden Client und der zu skalierenden Anwendung ist ein Proxy geschaltet, der als Loadbalancer agiert. Der Proxy nutzt ebenfalls einen Polling-Ansatz um die Routeninformationen von gestarteten Instanzen bei den CSPs zu erfragen. Anfragen, die den Proxy durchlaufen, werden nun automatisch zwischen lokalen und externen Instanzen einer Anwendung geroutet.

\subsection{Datenmodell}

Die Entitaeten sind in Fig. \ref{datenmodell} dargestellt. Eine Anwendung besteht aus mehreren Services, wobei ein Service das Element einer automatischen Skalierung durch den Autoscaler sein soll. Diese Services werden in einer Konfigurationsdatei (vgl. \ref{configuration}) als sogenannte \textit{ScrapeTargets} definiert. Ein ScrapeTarget beschreibt einen Service mit einer Metrikabfrage, die in Intervallen an das Monitoring-Tool gesendet und fuer die Berechnung der Skalierung eines Services evaluiert wird. Auch wird beschrieben, wie die zu startende Container-Instanz bei einem CSP parametrisiert (\textit{InstanceSpec}) wird und welche Instanzen in der lokalen Infrastruktur ausgefuehrt werden (\textit{staticSpec}).

\begin{figure}
	\centering
	\includegraphics[width=1.0\linewidth,scale=0.8]{images/datamodel.png}
	\caption{Entitaeten der Autoscaling-Architektur, dargestellt als Baumdiagramm}
	\label{datenmodell}
\end{figure}

Einem Service (\textit{ScrapeTarget}) sind mehrere Instanzen (\textit{Instances}) zugeordnet. Eine Instanz beschreibt einen zustandsbehafteten Prozess, in dem die zu skalierende Anwendung oder Teilanwendung ausgefuehrt wird oder ausgefuehrt werden soll. Die Parameterbeschreibung eines Containers  in einem ScrapeTarget,  dient als Schablone fuer das Erstellen von Instanzen die einem Service zugeordnet sind.

\begin{figure}
	\centering
	\includegraphics[width=0.6\linewidth,scale=0.6]{images/state.png}
	\caption{Zustandsdiagramm fuer Instanzen}
\end{figure}

Instanzen besitzen einen Lebenszyklus, welcher durch die Bearbeitung durch die Agent-Komponente veraendert wird. Diese Statusveraenderungen werden als \textit{Status} in einer Instanz vermerkt.
	
\subsection{Control Plane} \label{control_plane}

Die Control-Plane ist das Kernelement der Autoscaling-Architektur. Fuer jeden zu skalierenden Service, wird in zeitlichen Intervallen die festgelegte Metrik von dem Monitoring-Tool abgefragt. Das Ergebnis dieser Abfrage wird in die Berechnung fuer den Bedarf neuer Instanzen einkalkuliert. Anhand des Bedarfes wird der Zustand der Control-Plane modifiziert. Ueber eine RESTful Webschnittstelle koennen andere Komponenten wie der Proxy und die Agents diesen Zustand abrufen und veraendern, beispielsweise wenn der Status einer Instanz aktualisiert wurde.

\begin{figure}[h]
	\centering
	\includegraphics[width=0.8\linewidth,scale=0.8]{images/autoscaler.png}
	\caption{Komponentendiagramm der Control-Plane Komponente}
\end{figure}

In der hier vorgestellten Implementierung wird Prometheus\footnote{https://github.com/prometheus} als Monitoring-Tool verwendet.  Prometheus zeichnet Echtzeitmetriken in einer Zeitreihendatenbank auf, ueber Webaufrufe von Anwendungenen abgefragt werden. Ausserdem bietet Prometheus eine funktionale Abfragesprache namens PromQL (Prometheus Query Language) an, mit der Clients Zeitreihendaten in Echtzeit selektieren und aggregieren koennen. Das Ergebnis eines Ausdrucks wird von der Control-Plane in zeitlichen Intervallen ueber eine Webschnittstelle von Prometheus konsumiert. \\

Als Datenbank wird die in Go geschriebene BoltDB\footnote{https://github.com/etcd-io/bbolt} als einfacher, schneller und zuverlässiger Key-Value Store verwendet. Ueber eine Adapter-Komponente (\textit{StateProvider}) und ein definiertes Interface lassen sich jedoch weitere Datenbanken anbinden.

\subsubsection{Konfiguration} \label{configuration} \hfill\\

Die Autoscaling-Architektur muss sich auf beliebige containerisierte Anwendungen anwenden lassen. Auch muessen SLOs und fuer die automatische Skalierung genutzte Metriken flexibel und zentral konfigurierbar sein. Die Architektur verwendet dafuer eine externes cloudburst.yaml Konfigurationsdatei. Ein Beispiel fuer eine cloudburst.yaml Datei ist in Abbildung \ref{cloudburst_config} dargestellt. Die .yaml Dateierweiterung nimmt bereits vorweg, dass es sich um eine in YAML strukturierte Textdatei handelt, die einem bestimmten Beshreibungsschema folgt. \\

\begin{lstlisting}[label={cloudburst_config}, caption={Ein Beispiel fuer eine \textbf{cloudburst.yaml} Datei. Definiert wird ein Service mit einer Abfrage formuliert in PromQL, einer Angabe der Parameter fuer zu startende Container-Instanzen (\textit{spec}) und eine Liste von lokal erreichbaren Endpunkten (\textit{static})},captionpos=b]
prometheus_url: http://localhost:9090
targets:
	- name: bubblesort-service
		path: /bubblesort
		description: requests amount greater than 150 rps
		query: |
		(sum(rate(example_sorting_requests_total[30s])) / 150)
		spec:
			container:
				name: "bubblesort-svc"
				image: cbrgm/example-app:latest
				...
		static:
			endpoints:
				- http://localhost:9997
\end{lstlisting}

Auf oberster Ebene wird die Adresse der Prometheus-Instanz definiert, von welcher Metriken ueber zu skalierende Services abgerufen werden. Darauf folgt eine Liste an Targets, wobei ein Target mit einem Service einer Anwendung gleichzusetzen ist. Ein Service wird definiert durch einen einzigartigen Namen, einem Pfad unter dem der Service durch den Proxy erreichbar ist und einer in PromQL formulierten Metrikabfrage. Ausserdem wird definiert, wie der Container einer neu gestarteten Instanz parametrisiert werden soll und welche Endpunkte einer Anwendung lokal erreichbar sind.



\subsubsection{Processor} \hfill\\

Die Prozessor-Komponente der Control-Plane ist verantwortlich fuer das Abfragen der Metrikwerte und fuer das Berechnen der Allokierung oder Freigabe von Ressourcen fuer die in der Konfigurationsdatei definierten Services. Die Berechnung wird hierbei delegiert an die Autoscaler-Komponente und das Allokieren und Freigeben wiederum wird von der Requester-Komponente uebernommen.
	
\subsubsection{Autoscaler} \label{autoscaler} \hfill\\

Die Autoscaler-Komponente berechnet die Allokierung oder Freigabe von Ressourcen. Bei jedem Evaluationsintervall $i$, wird auf Basis des Ergebnisses der zu einem Service zugehoerigen Metrikabfrage $q_{i}$ und dem Zustand der Control-Plane $S_{i}$ zu einem bestimmten Zeitpunkt $t_{i}$ ein Bedarf an Ressourcen $\Delta _{i}$ berechnet . Der Zustand $S_{i}$ ist definiert als

\begin{gather*}
	S_{i}=\left( I_{(in,r)_{i}}, I_{(ex,r)_{i}}, I_{(ex,pr)_{i}},  I_{(ex,md)_{i}} , q_{i} \right)
\end{gather*}

wobei

	$I_{(in,r)}$ = Instanzen \textit{intern} mit Zustand \textit{Running} \\
	$I_{(ex,r)}$ = Instanzen \textit{extern} mit Zustand \textit{Running} \\
	$I_{(ex,pr)}$ = Instanzen \textit{extern} mit Zustand \textit{Progress} \\
	$I_{(ex,md)}$ = Instanzen \textit{extern} mit markierter Loeschung \\
	$q$ = Ergebnis einer Metrikabfrage
	
sowie der Bewertungsfunktion  $B$ fuer die Zustand $S_{i}$. \\

$\Delta _{i}$ berechnet sich aus dem Zustand $S_{i}$ und der Bewertungsregeln $B(S_{i})$. Die Bewertungsregeln die fuer das automatische Skalieren von Ressourcen verwendet werden koennen, sind Faktoren die sich aus Performanz-, Kosten- oder Verfuegbarkeitszielen ergeben.\\

Die Variablen des Zustandvektors $S_{i}$ werden fuer interne Ressourcen aus den Angaben in der Konfigurationsdatei und fuer externe Ressourcen anhand des Status der Instanzen zur Laufzeit ermittelt. Die Metrikabfrage $q$ wird ebenfalls in der Konfigurationsdatei (vgl. \ref{configuration}) definiert. Die Berechnung von $\Delta _{i}$ der in dieser Arbeit vorgestellten Architektur unter Verwendung der beschriebenen Variablen wird in Abschnitt \ref{skalierungstechnik} dargestellt. 
	
\subsubsection{Requester}  \hfill\\

Die Requester-Komponente ist fuer die Veraenderung des Zustandes der Control-Plane verantwortlich. Anhand des berechneten Bedarfes durch die Autoscaler-Komponente, werden Datenstrukturen von Instanzen erzeugt oder bestehende Datenstrukturen zur Terminierung durch die Agents markiert.

\begin{algorithm}[H]
	\DontPrintSemicolon
	\KwIn{demand calculation result $d$}
	\KwData{list of instances $I$}
	 \If{$d$ == 0}
	{
		get list of \textit{pending} instances $i_{p} \subseteq I$ \;
		set $I = I \setminus i_{p}$ \;
	}
	
	\texttt{\\}
	\ElseIf{$d$ \textgreater 0}
	{
		get list of \textit{pending} instances $i_{p} \subseteq I$ \;
		set $d = d - length(i_{p})$ \;
		
		\If{$d$ \textgreater 0}
		{
			
			\While{length($i_{p}$) \textless $d$}
			{
				add new \textit{pending} instance to  $i_{p}$
			}
			
		}
	
		\If{$d$ \textless 0}
		{
			remove $|d|$ \textit{pending} instances $i_{p} \in I$
		}
	}
	
	\texttt{\\}
	\ElseIf{$d$ \textless 0}
	{
		get list of \textit{pending} instances $i_{p} \subseteq I$ \;
		set $I = I \setminus i_{p}$ \;
		get list of \textit{running} $\wedge$ \textit{not active} instances $i_{r} \subseteq I$ \;
		\If{$|d|$ \textgreater= 0}
		{
			mark all $i \in i_{r}$ as \textit{not active}
		}
		\Else 
		{
			mark $|d|$ instances from $i_{r}$ as \textit{not active}
		}
	}
	
	\caption{resource provisioning based on instance state}
	\label{fig:requester_algorithm}
\end{algorithm}

Die Funktionsweise des Requesters ist in Alg. \ref{fig:requester_algorithm} dargestellt. In Versuchen wurde festgestellt, dass bei fast gleichbleibender Lastverhalten, Instanzen gestartet und gestoppt wurden. Diesem wird entgegengewirkt durch die Einfuehrung eines Thresholds, ab dem die Steuerung der Provisionierung reagiert. Dadurch  wird die Steuerung  traeger und somit unempfindlicher gegenueber Schwankungen in Ergebnissen de Messintervalle. \\


\begin{algorithm}[H]
	\DontPrintSemicolon
	\KwIn{demand calculation result $d$, threshold $t$}
	\KwOut{demand calculation result $d$}
	\If{$d >= t_{min} \wedge d <= t_{max}$}
	{
		return 0
	}
	return $d$
	
	\caption{threshold calculation}
\end{algorithm}

\subsection{Agents}

Die Agents sind verantwortlich fuer das Allokieren und Freigeben von Ressourcen bei einem CSP gemaess des Zustands der Control-Plane. Die Agents sind zustandslos und fuehren in festgelegten Intervallen die in Alg. \ref{agent_algorithm} vorgestellte Routine aus. \\

\begin{algorithm}[H]
	\DontPrintSemicolon
	agent initialization\;
	\While{running}{
		query list of scrapeTargets $S$ from control plane \;
		\ForEach {$s \in \mathcal S $}
		{
			query list of instances $I$ for $s$ from control plane \;
			get list of  \textit{pending} instances $i_{p} \subseteq I$ \;
			get list of  \textit{not active} instances $ i_{t} \subseteq I$ \;
			\ForEach {$i \in i_{p} \cup i_{t}$}
			{
				set status of $i$ as \textit{progress}
			}
		
			update status of {$i \in i_{p} \cup i_{t}$} in control plane \;
			\ForEach {$i \in i_{p}$}
			{
				provision instance at  CSP \;
				\If{success}{set status of $i$ as \textit{running}}
				\Else{set status of $i$ as \textit{failure}}
			}
			\ForEach {$i \in i_{t}$} {
				terminate instance at CSP \;
				set status of $i$ as \textit{terminated}
			}
			update status of {$i \in i_{p} \cup i_{t}$} in control plane \;
		}
	
		
		
		
	}
	\caption{agent resource provisioning routine}
	\label{agent_algorithm}
\end{algorithm}

Fuer jeden genutzten CSP wird genau ein Agent gestartet. Ueber ein definiertes Interface werden plattformspezifische Schnittstellen zur Provisionierung von Ressourcen bei dem CSP angebunden. Ein Agent kann auf der eigenen lokalen Infrastruktur, oder aber direkt auf der Plattform des CSP ausgefuehrt werden.
	
\subsection{Proxy}

Damit die Anfragen eines Klienten zwischen laufenden Instanzen auf der lokalen Infrastruktur und provisionierten Instanzen bei einem CSP verteilt werden koennen, wird ein Proxy in Form eines Loadbalancers zwischen der zu skalierenden Anwendung und den Klienten platziert. \\

Die Proxy-Komponente fragt in Intervallen die Control-Plane nach aktiven Instanzen ab, um anschliessend die aktiven Endpunkte fuer die Lastverteilung zu konfigurieren. Die vorgestellte Architektur verwendet als Loadbalancer das in Go geschriebene Open-Source Projekt Skipper\footnote{https://github.com/zalando/skipper} von der Firma Zalando. ergaenzt um eine eigene DataProvider Implementierung, um aktive Instanzen von der Control-Plane-Komponente abzufragen und in eine Routenkonfiguration fuer Skipper zu uebersetzen.

\subsection{Skalierungstechnik} \label{skalierungstechnik}

Die in der Arbeit realisierte Skalierung misst an Hand einer anwendungsspezifischen Metrik (siehe Metrikabfrage $q$ in Abschnitt \ref{autoscaler}), in welchem Verhaeltnis $V$ die benoetigte Zahl der Instanzen zu der aktuellen Zahl der Instanzen steht. Eine Annahme der  Berechnungen des Ressourcenbedarfes ist, dass die Leistung von internen und extern allokierten Instanzen in etwa gleich ist.  \\

$\Delta_{i}$ ist die Zahl, wieviele Instanzen zu den aktuell Existierenden zusaetzlich allokiert oder geloescht werden muessen. Sie berechnet sich als Produkt aus $(V-1)$ und der Summe der aktiv laufenden Instanzen $(i_{in,r} + i_{ex,r})$. Wenn sich die bereits parallel in Erstellung befindlichen Instanzen $i_{ex,pr}$ und zur Loeschung vorgemerkten Instanzen $ i_{ex,md}$ ergibt sich die Formel
\begin{center}
	$\Delta_{i} = (V-1) \times (i_{in,r} + i_{ex,r}) - i_{ex,pr} + i_{ex,md}$
\end{center}

Da das Ergebnis fuer den Ressourcenbedarf ganzzahlig sein muss, wird das nächsthöhere Ganzzahlige von $\Delta_{i}$ von der Skalierungsfunktion an die Requester-Komponente in der Control-Plane übermittelt.

% file:///home/chris/Zotero/storage/LUKLJ8WX/Gandhi%20et%20al.%20-%20Adaptive,%20Model-driven%20Autoscaling%20for%20Cloud%20Appli.pdf

\section{Experimente}

In diesem Abschnitt wird erlaeutert, wie die Autoscaling-Architektur eine containerisierte Anwendung unter verschiedenen Lastverlaeufgen skalieren kann. Hierzu wurde eine Beispielanwendung implementiert und die automatische Skalierung mit unterschiedlichen Anfrageszenarien getestet. Der Testaufbau, die Durchfuehrung und die Ergebnisse werden im Folgendem detailiert beschrieben.

\subsection{Testumgebung}

Die Control-Plane, der Agent, der Proxy und eine zu skalierende Beispielanwendung, welche in \ref{Beispielanwendung} genauer beschrieben wird, wurden auf einer Testinstanz der Infrastruktur der HAW Hamburg in Containern ausgefuehrt. Die Instanz besitzt einen E5-2670-Prozessor und 32 GB RAM und ist ueber Intel Gigabit Ethernet angebunden. Als Betriebssystem wurde Ubuntu 18.04 verwendet mit der Linux-Kernel Version 5.3. \\

In der Agent-Komponente wurde ein Mock implementiert, um einen externen CSP zu simulieren. Der Agent provisioniert keine "echten" Ressourcen, sondern simuliert das Erstellen von neuen Ressourcen durch eine Wartezeit zwischen 3 bis 5 Sekunden pro Instanz. Saemtliche Fake-Instanzen verweisen auf einen separat gestartete "Sink"-Anwendung, die dasselbe Interface wie die in Abschnitt \ref{Beispielanwendung} vorgestellte Beispielanwendung besitzt, aber statt Anfragen zu verarbeiten diese verwirft. Hierdurch wird die Last der Beispielanwendung reduziert und eine Skalierung von Ressourcen simuliert.

\subsubsection{Beispielanwendung} \label{Beispielanwendung} \hfill\\

Als Beispielsanwendung fuer die automatische Skalierung, wurde ein Microservice\footnote{https://github.com/cbrgm/cloudburst/example} implementiert, der ueber eine RESTful Webschnittstelle die Sortierung von Zahlenreihen durch eine Bubblesort-Implementierung anbietet. Da der Bubblesort-Algorithmus eine durschnittliche Laufzeit von $\Theta \left( n^{2}\right)$ besitzt, kann durch die Erhoehung der Zahlenreihe in einer Anfrage der Aufwand beliebig konfiguriert werden. Ein Container-Image wurde fuer den Microservice erstellt und in der Container-Registry der Gitlab-Instanz der HAW Hamburg bereitgestellt. \\

Desweiteren wurden Metriken wie z.B. Bearbeitungsdauer und Anzahl von HTTP-Anfragen in die Anwendung implementiert und koennen zur Laufzeit unter einem /metrics Endpunkt ueber den Service abgerufen werden.

\subsection{Durchfuehrung}

Als Skalierungsregel fuer die Bubblesort-Service Instanzen wurde festgelegt, dass die durchschnittlichen HTTP-Anfragen pro Sekunde einen Threshold von $th_{max}$ = 150 nicht ueberschreiten duerfen d.h. der Service  wird skaliert, sobald die Instanzen im Durchscnitt mehr als 150 HTTP-Anfragen pro Sekunde erhalten. \\

Dafuer wurde die anwendungsspezifische Metrik $q$ = durchschnittliche HTTP-Anfragen pro Sekunde verwendet.
Das in in Abschnitt \ref{autoscaler} beschrieben Verhaeltnis V wird durch den Ausdruck $(q/th_{max})$ berechnet und ergibt damit als $\Delta_{i}$

\begin{center}
	$\Delta_{i} = (q/th_{max}-1) \times (i_{in,r} + i_{ex,r}) - i_{ex,pr} + i_{ex,md}$
\end{center}

Zur Simulation von Benutzeranfragen wurde das Tool Artillery\footnote{https://artillery.io/} verwendet. Die Lastszenarien wurden in Artillerys eigenem Konfigurationsformat beschrieben, welches  in der offiziellen Dokumentation\footnote{https://artillery.io/docs/} des Tools nachlesbar ist. Die Skriptingmoeglichkeiten von Artillery in Form von Javascript Callbacks  wurden genutzt, um fuer jede simulierte Benutzeranfrage an den Bubblesort-Service eine zufaellige Zahlenreihe mit einer festen Laenge von 5000 Elementen zu generieren. Jedes Lastszenario besitzt eine Laufzeit von insgesamt 10 Minuten mit variablen Anfrageverlaeufen.\\

Die vorgestellte Autoscaling-Architektur ist ein verteiltes System, dass die automatische Skalierung von Ressourcen asynchron umsetzt. Deshalb gibt es mehrere Faktoren, die Einfluss auf die Skalierung der Ressourcen haben. Folgende Zeitintervalle fuer die Abfragen zwischen den Komponenten der verteilten Autoscaling-Architektur wurden fuer die Experimente verwendet: 

 \begin{itemize}
	\item  5 Sek. Zeitintervall fuer das Abfragen von Metriken zwischen dem Bubblesort-Service und Prometheus
	\item  20 Sek. Zeitintervall fuer das Abfragen von Metriken und Berechnung des Ressourcenbedarfs zwischen Prometheus und der Control-Plane
	\item  5 Sek. Zeitintervall fuer das Abfragen des Zustands der Control-Plane durch die Agents
\end{itemize}

Zuvor wurden Versuche mit unterschiedlichen Zeitintervallen zwischen den Komponenten vorgenommen, um eine geeignete Konfiguration zu ermitteln.
Das Zeitintervall zwischen der Control-Plane und den Agents muss wesentlich unter dem Zeitintervall fuer das Abfragen von Metriken und Berechnung des Ressourcenbedarfs zwischen Prometheus und der Control-Plane liegen. Ansonsten findet zwischen zwei Berechnung des Ressourcenbedarfs durch die Control-Plane keine Allokierung oder Freigabe von Ressourcen durch die Agents statt. Das Zeitintervall fuer Berechnung des Ressourcenbedarfs durch die Control-Plane darf nicht zu kurz gesetzt werden, da sich ansonsten die Wirkung der Skalierung nicht in den fuer die Skalierung genutzten Metriken des Services wiederspiegelt und es hierdurch kurzzeitig zu einer Uebersteuerung kommt. \\

\subsubsection{Workload-Profile} \hfill\\

Als Lastszenarien wurden vier typische Anfrageverlaeufe in Anwendungssystemen definiert. Alle Lastszenarien sind im Projektordner\footnote{https://github.com/cbrgm/cloudburst/experiments} unter "Experiments" zu finden.

 \begin{itemize}
 	\item \textbf{Cold to Spike} - Dieses Lastszenario simuliert einen Anfrageverlauf, bei dem Anfragen zwischen zwei Extremen schwanken. Zu einem Zeitpunkt werden fast keine Anfragen an die Anwendung gestellt, waehrend wenig spaeter die Anfragen ohne lange Wachstumsphase, rapide auf ein Maximum ansteigen. Das Lastszenario wird durch die Datei \textit{workload-cold-to-spike.yaml} beschrieben.
 	
 	\item \textbf{Periodic Spikes} - In diesem Lastszenario wird ein Anfrageverlauf definiert, der konstant verlaeuft, aber in festen, wiederkehrenden Intervallen einen Anstieg von Anfragen mit Wachstumsphase aufweist.  Das Lastszenario wird durch die Datei \textit{workload-periodic-spikes.yaml} beschrieben.
 	
 	\item \textbf{Random Spikes} - Der Anfrageverlauf dieses Lastszenarios aehnelt dem des \textit{Periodic Spikes}, mit den Unterschied dass das Anfragewachstum unterschiedlich stark und die Wachstumsintervalle unterschiedlich lange ausfallen.  Das Lastszenario wird durch die Datei \textit{workload-random-spikes.yaml} beschrieben.
 	
 	\item \textbf{Rapid Growth} -  Dieses Lastszenario simuliert einen Anfrageverlauf, der zunaechst konstant waechst, eine Plateauphase erreicht und anschliessend langsam wieder schrumpft. Das Lastszenario wird durch die Datei \textit{workload-rapid-growth.yaml} beschrieben.
 \end{itemize}

\subsection{Auswertung}

Um das Verhalten der Autoscaling-Architektur zu ueberpruefen wurde direkt nach der Durchfuehrung eines Lastszenarios eine Zeitreihe mit Messwerten von verschiedenen Metriken exportiert. Darunter die durchschnittliche Anzahl der HTTP-Aufrufe an der Proxy-Komponente, um durch das Teilen durch den zu erreichenden Threshold einen exakten Bedarf zu einem Zeitpunkt berechnen zu koennen. Dem gegenueber steht die Anzahl der der laufenden Instanzen, provisioniert durch die Agents und aufgezeichnet vor einem Berechnungsschritt der Control-Plane. Ausserdem wurde die durchschnittliche Anzahl der HTTP-Aufrufe an dem zu skalierenden Bubblesort-Service gemessen, um nachzuvollziehen, ob die Anzahl der Aufrufe durch die Skalierung von Instanzen unter dem festgelegten Threshold bleibt.

\subsection{Ergebnisse}

Im Folgenden werden die Ergebnisse der durchgefuehrten Lastszenarien beschrieben. Bei den Ergebnissen jeder Durchfuehrung ist zu beobachten, dass die Anzahl der aktiven Instanzen leicht verzoegert gegenueber der berechneten, benoetigten Anzahl an Instanzen ist. Grund hierfuer ist die Entscheidung fuer den Einsatz der reaktiven Skalierungstechnik. Die Anzahl der aktiven Instanzen in den Auswertungskurven des Skalierungsverhaltens faellt nicht unter den Wert von einer Instanz, da die lokal laufende Instanz des Bubblesort-Services in der Darstellung schon mit beruecksichtigt wurde. \\

\begin{figure}
	\centering
	\begin{subfigure}{.5\textwidth}
		\centering
		\includegraphics{figures/workload-cold-to-spike-01.pdf}
		\label{fig:sub11}
	\end{subfigure}%
	\begin{subfigure}{.5\textwidth}
		\centering
		\includegraphics{figures/workload-cold-to-spike-02.pdf}
		\label{fig:sub2}
	\end{subfigure}
	\caption{Benchmark: workload-cold-to-spike.yaml}
	\label{fig:benchmark_01}
\end{figure}

In Abb. \ref{fig:benchmark_01} ist zu beobachten, dass die durchschnittliche Anzahl an HTTP-Anfragen pro Sekunde an den Bubbelsort-Service zu Beginn jeder Wachstumsphase den Threshold stark uebersteigen. Grund hierfuer ist, dass die Control-Plane in Intervallen agiert. Zu beobachten ist, dass ab der 100 Sekunde eine Berechnung des Ressourcenbedarfes durch die Control-Plane durchgefuehrt und wenig  spaeter erfolgreich zwei Instanzen zusaetzlich durch den Agent provisioniert wurden. Die Wirkung der Lastverteilung durch den Proxy zeigt sich wenig spaeter durch das Abfallen der Lastspitze unter den Zielwert des Thresholds von 150 durchschnittlichen  HTTP-Anfragen pro Sekunde. Interessant zu beobachten ist der Aussreisser bei Sekunde 330. Hier wird eine Berechnung des Ressourcenbedarfs durchgefuehrt und in Folge eines ueberstiegenden Thresholdwertes zwei Instanzen provisioniert. Jedoch ist der Thresholdwert zum Zeitpunkt der folgenden Berechnung des Ressourcenbedarfs trotz des starken Abflachens der Anfragen nach wie vor ueber dem Zielwert von 150, weshalb kurzzeitig eine dritte Instanz provisioniert wird. Die dritte Instanz wird wenig spaeter freigegeben, da sich nun der Thresholdwert unter dem Zielwert befindet. \\
	
\begin{figure}
	\centering
	\begin{subfigure}{.5\textwidth}
		\centering
		\includegraphics{figures/workload-periodic-spikes-01.pdf}
	\end{subfigure}%
	\begin{subfigure}{.5\textwidth}
		\centering
		\includegraphics{figures/workload-periodic-spikes-02.pdf}
	\end{subfigure}
	\caption{Benchmark: workload-periodic-spikes.yaml}
	\label{fig:benchmark_02}
\end{figure}	

In Abb. \ref{fig:benchmark_02} ist ein aehnliches Verhalten wie in  \ref{fig:benchmark_01} zu beobachten. Zu beobachten ist, wie zusaetzliche Instanzen provisioniert werden, sobald der Thresholdwert von 150 ueberschritten wird. Zum Ende der beiden Wachstumsphasen bei Sekunde 200 und 400 ist ein kurzer Anstieg der Instanzenanzahl zusehen. Anhand der durchschnittlichen Anzahl an HTTP-Anfragen pro Sekunde an den Bubbelsort-Service ist zu erkennen, dass der der Thresholdwert von 150 erneut ueberschritten wurde und in Folge  kurzzeitig eine weitere Instanz gestartet wurde. Die Anzahl der laufenden Instanzen sinkt wenig spaeter wieder auf den Anfangswert von 1, da auch die Lastspitze stark abfaellt. Dieses Verhalten ist auch in Abb. \ref{fig:benchmark_03} zusehen.

\begin{figure}
	\centering
	\begin{subfigure}{.5\textwidth}
		\centering
		\includegraphics{figures/workload-random-spikes-01.pdf}
	\end{subfigure}%
	\begin{subfigure}{.5\textwidth}
		\centering
		\includegraphics{figures/workload-random-spikes-02.pdf}
	\end{subfigure}
	\caption{Benchmark: workload-random-spikes.yaml}
	\label{fig:benchmark_03}
\end{figure}

In Abb. \ref{fig:benchmark_04} ist deutlich zu erkennen, wie die Autoscaling-Architektur mit steigender Anfragekurve die Anzahl der benoetigten Instanzen adjustiert. Ab Sekunde 100 wird der Thresholdwert von 150 ueberstiegen und eine weitere Instanz gestartet. Die durchschnittliche Anzahl an HTTP-Anfragen pro Sekunde an den Bubbelsort-Service flacht daraufhin kurzzeitig ab, ehe sie langsam wieder ansteigt und erneut den Thresoldwert uebersteigt. Anschliessend wird reagiert und wieder eine weitere Instanz gestartet. Waehrend der Plateauphase beginnend ab Sekunde 210 ist deutlich zu erkennen, dass das Ziel die durchschnittliche Anzahl an HTTP-Anfragen pro Sekunde fuer eine Instanz unter einem Wert von 150 zu halten, erfuellt wird. Durch das Abfallen des Anfrageverlaufes wird ab Sekunde 400 eine Instanz abgeschaltet, in Folge dessen erhaelt der Bubblesort-Service mehr Anfragen die zu Bearbeiten sind. Dem Thresholdwert wird sich angenaehert und kurzzeitg ueberschritten. Jedoch viel das Berechnungsintervall fuer den Ressourcenbedarf nicht in den Zeitpunkt der Ueberschreitung, weshalb keine weitere Instanz gestartet worden ist. Zum Ende des Lastszenarios faellt die Anfragekurve langsam ab und die Anzahl der laufenden Intanzen reduziert sich erneut auf den Anfangswert von 1. \\

\begin{figure}
	\centering
	\begin{subfigure}{.5\textwidth}
		\centering
		\includegraphics{figures/workload-rapid-growth-01.pdf}
		\label{fig:sub1}
	\end{subfigure}%
	\begin{subfigure}{.5\textwidth}
		\centering
		\includegraphics{figures/workload-rapid-growth-02.pdf}
		\label{fig:sub2}
	\end{subfigure}
	\caption{Benchmark: workload-rapid-growth.yaml}
	\label{fig:benchmark_04}
\end{figure}

Zusammenfassend sind die Ergebnisse sehr zufriedenstellend. Es zeigt sich, dass die Autoscaling-Architektur als verteiltes System in allen durchgefuehrten Lastszenarien in der Lage ist, dynamisch auf den Ressourcenbedarf zu reagieren.

\section{Zusammenfassung und Ausblick}

Ziel dieser Arbeit war die Konzeptionierung, Implementierung und Evaluierung einer Autoscaling-Architektur fuer Cloud-Bursting in Hybrid-Clouds, die automatisiert Ressourcen auf Basis von Metriken als Container-as-a-Service bei mehreren CSPs provisionieren und skalieren kann. Hierfuer wurden zwei wichtige Beitraege geliefert. Zum einen eine konkrete Systemarchitektur fuer einen verteilten Autoscaler und zum Anderen eine Evaluation eines Prototypen, welcher durch eine reaktive Skalierungstechnik die den Lebenszyklus von Ressourcen beruecksichtigt, dynamisch auf unterschiedliche Lastszenarien reagieren kann. \\

Es wurde gezeigt, wie sich die automatische Skalierung von Ressourcen fuer on-premise betriebene Services einer zusammenhaengenden Anwendung realisieren laesst, ohne das hierfuer eine Cloud-Infrastruktur betrieben werden muss und wie sich dynamisch provisionierte Ressourcen bei CSP in die eigene Infrastruktur integrieren lassen, ohne das ein hoher Konfigurationsaufwand fuer den Infrastrukturbetrieb ensteht. Die in dieser Arbeit vorgestellte verteilte Autoscaling-Architektur ist offen fuer Erweiterungen, sodass sich weitere Skalierungstechniken implementieren und in einem verteilten Kontext evaluieren lassen. \\

In zukuenftigen Arbeiten soll an der Verbesserung des Prototypen gearbeitet werden. Dafuer sollen weitere Skalierungstechniken evauliert, die Cloudplattformen verschiedener CSPs angebunden und die Architektur mit weiteren anwendungsspezifischen Metriken getestet werden.
	
% ---- Bibliography ----
%
% BibTeX users should specify bibliography style 'splncs04'.
% References will then be sorted and formatted in the correct style.
%
\newpage
\bibliographystyle{splncs04}
\bibliography{paper.bib}
%
\end{document}
